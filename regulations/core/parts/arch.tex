\section{教学部}
    教学部分为备课组、授课组及竞赛组。其中备课组与授课组的设计目的为将材料准备与课堂讲解分离,发挥教学部成员的长处,同时备课的资料暂定存在 \href{https://github.com/iDataHub}{GitHub/iDataHub},以便资料的积累。竞赛组负责组织 Hackthon 竞赛,同时跟进进度及竞赛后续事项。所有组均需对接公共关系管理部宣传运营组及行政部财务组。
    \subsection{备课组}
    进行备课,积累完善讲义。对接项目部,安排 bi-week 的项目进度跟进和科普主题确定。


    \subsection{授课组}
    编程,算法,工具入门教学。


    \subsection{竞赛组}
    Hackthon,竞赛组织及评审。



\section{项目部}
    项目部管理 DataHub 的所有项目,需跟进项目立项及交付,同时需对接教学部、公共关系需管理部及行政部财务组。
    \subsection{项目管理组}
    跟进项目进度、对接教学(workshop)、对接财务。



\section{行政部}
    教学部分为办公组、财务组、法务组及运维组。
    \subsection{办公组}
    社团日常事务运营,如社团成立文件、招新、视频录制等。


    \subsection{财务组}
    发票管理及报销,项目组财务相关事务。


    \subsection{法务组}
    法律咨询。


    \subsection{运维组}
    负责管理管理 VPS,团队域名解析。




\section{公共关系管理部}
    教学部分为外联组及宣传运营组。
    \subsection{外联组}
    负债社团对外事务,如联系项目、寻找赞助等。


    \subsection{宣传运营组}
    对外宣传,包括招新、教学活动、项目成员招募等。
