\section{社团纳新}
    \subsection{纳新要求}
    \begin{itemize}
        \item 认同社团的核心价值观;
        \item 有强烈的服务精神,若有一定的社团及学生组织的管理经验更佳;
        \item 特殊职能部门,比如运维组,需要有特定技术背景。
    \end{itemize}


    \subsection{社团纳新途径}
    社团纳新主要通过以下两个途径进行。
    \begin{itemize}
        \item \textbf{校园纳新:} 通过每学期(年)的校内招新宣讲会,从常规通道加入社团。
        \item \textbf{内部推荐:} 通过内部成员推荐,通过快速通道加入社团。
    \end{itemize}


    \subsection{纳新通道}
    \begin{itemize}
        \item \textbf{常规通道:} 两轮筛选:简历及面试。
        \item \textbf{快速通道:} 直接面试,需至少有一名 Board 成员在场及批准。
    \end{itemize}



\section{项目纳新}
    \subsection{纳新要求}
    \begin{itemize}
        \item 认同数据科学的价值以及社团的核心价值观。
        \item 有与社团现有成员相当的学术及职业背景。
            \begin{itemize}
                \item \textbf{统计:} 先修如统计线性模型,贝叶斯分析,计算统计,大数据导论等课程。
                \item \textbf{计算机:} 先修如数据结构,机器学习,算法等课程。
                \item \textbf{职业背景:} 有实习或课程项目经验。
            \end{itemize}
        \item 兼备特别专长/专业方向的同学优先录取。
    \end{itemize}


    \subsection{项目纳新途径}
    项目纳新主要通过以下三个途径进行。
    \begin{itemize}
        \item \textbf{社团内部:} 通过内部成员推荐,通过快速通道加入社团。
        \item \textbf{校园纳新:} 通过每学期(年)的校内招新宣讲会,从常规通道加入社团。
        \item \textbf{Hackthon 竞赛} Hackthon 竞赛中的优秀参赛者可以直通社团项目组。
    \end{itemize}


    \subsection{纳新通道}
    \begin{itemize}
        \item \textbf{常规通道:} 两轮筛选:简历+面试。
        \item \textbf{快速通道:} 直接面试,需至少有一名 Board 成员在场及批准。
    \end{itemize}



\section{社团换届}
每届社团管理层包括社长、部长(副社长)、组长(副部长),任期为一学年(两学期)。原则上,社长及部长有同届的副社长及副部长继任。社团管理层以全社成员的加权投票产生。具体加权规则待确定。
